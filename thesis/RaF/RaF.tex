\chapter{\label{ch:req_and_specs} Requirements and Specifications}

As outlined in Section \ref{sec:objectives}, the system needs to meet specific requirements to ensure it performs as intended. These requirements were established through research, discussions with supervisor, and analysis of the intended user environment.

\begin{table}[h]
	\centering
	\begin{tabular}{|p{0.9cm}|p{6cm}|p{0.9cm}|p{6cm}|}
		\hline
		\textbf{Req. ID} & \textbf{Requirement Description} & \textbf{Spec. ID} & \textbf{Specification Description} \\ \hline
		\textbf{R1} & The robot must operate in different control zones, defined by AR markers, and adjust its behavior accordingly (e.g., slowing down, stopping, or performing a task). & \textbf{F1} & Implementation of AR-based control zones that dynamically adjust robot speed, direction, or initiate tasks based on detected markers. \\ \hline
		\textbf{R2} & The system must integrate dynamic instructions based on ArUco marker detection, allowing for real-time task updates. & \textbf{F2} & Real-time processing of ArUco markers to update task instructions and provide context-specific navigation and task execution. \\ \hline
		\textbf{R3} & The AR interface must provide real-time visual feedback for the human operator to improve task monitoring and control. & \textbf{F3} & AR-based visual feedback will provide the human operator with real-time monitoring of the robot’s actions and environmental interactions. \\ \hline
		\textbf{R4} & The robot must include object avoidance functionality, ensuring it maintains a safe distance from recognized obstacles (e.g., warning cones) based on AR markers. & \textbf{F4} & Integration of object avoidance behavior, ensuring the robot maintains a safe distance from obstacles based on AR marker detection. \\ \hline
		\textbf{R5} & The system should detect and respond to wheel slip events caused by changes in surface traction, providing stability during operation. & \textbf{F5} & Implementation of wheel slip detection functionality that monitors the robot's power draw to identify surface traction changes. \\ \hline
	\end{tabular}
	\caption{Requirements and Functionalities for the Robot Control System}
	\label{tab:requirements_functionalities}
\end{table}


\section{Acceptable Test Procedures}

Testing will be performed to ensure that the system meets the outlined requirements and functionalities. The table below shows the breakdown of the sub-tests to be performed, with each test linked to the specific functions and requirements it checks.

\begin{table}[ht]
	\centering
	\caption{Enhanced Breakdown of sub-tests to be performed in the acceptance testing.}
	\label{tab:testing}
	\begin{tabular}{|p{0.8cm}|p{3.6cm}|p{7.5cm}|p{0.9cm}|p{0.9cm}|}
		\hline
		\textbf{Test} & \textbf{Description} & \textbf{Expected Outcome} & \textbf{Spec. ID} & \textbf{Req. ID} \\ \hline
		T1 & Test the robot’s behavior in AR-based control zones & The robot adjusts speed or direction based on AR markers within defined zones (e.g., slow zone, stop zone). & F1 & R1 \\ \hline
		T2 & Test real-time ArUco marker detection for dynamic instructions & Robot recognizes ArUco markers in real-time and triggers corresponding task instructions (e.g., task initiation or environmental change response). & F2 & R2 \\ \hline
		T3 & Test AR-based visual feedback for real-time monitoring & The user receives accurate and real-time AR feedback on the robot’s position, detected markers, and current task. & F3 & R3 \\ \hline
		T4 & Test object avoidance system using visual markers & The robot successfully avoids predefined obstacles marked by ArUco codes within 20 cm, halting or rerouting its path. & F4 & R4 \\ \hline
		T5 & Test wheel slip detection and response functionality & The robot detects slip on different surfaces (e.g., sand or wet surfaces) and compensates by adjusting motor power or stopping. & F5 & R5 \\ \hline
	\end{tabular}
\end{table}



